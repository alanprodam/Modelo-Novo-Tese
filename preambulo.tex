%ARQUIVO DE PREAMBULO DA TESE - PACOTES E CONFIGURA\c{C}\~{O}ES

\documentclass[
	% -- op\c{c}\~{o}es da classe memoir --
	12pt,				% tamanho da fonte
	%openright,			% cap\'{\i}tulos come\c{c}am em p\'{a}g \'{\i}mpar (insere p\'{a}gina vazia caso preciso)
	oneside,			% para impress\~{a}o em verso e anverso. Oposto a oneside
	a4paper,		% tamanho do papel.
	% -- op\c{c}\~{o}es da classe abntex2 --
	chapter=TITLE,		% t\'{\i}tulos de cap\'{\i}tulos convertidos em letras mai\'{u}sculas
	%section=TITLE,		% t\'{\i}tulos de se\c{c}\~{o}es convertidos em letras mai\'{u}sculas
	%subsection=TITLE,	% t\'{\i}tulos de subse\c{c}\~{o}es convertidos em letras mai\'{u}sculas
	%subsubsection=TITLE,% t\'{\i}tulos de subsubse\c{c}\~{o}es convertidos em letras mai\'{u}sculas
	% -- op\c{c}\~{o}es do pacote babel --
%	english,			% idioma adicional para hifeniza\c{c}\~{a}o
	%french,			% idioma adicional para hifeniza\c{c}\~{a}o
	%spanish,			% idioma adicional para hifeniza\c{c}\~{a}o
	brazil,				% o \'{u}ltimo idioma \'{e} o principal do documento
	english,
	sumario=tradicional,
	]{abntex2}


% ---
% PACOTES
% ---

% ---
% Pacotes fundamentais
% ---

\usepackage{chngcntr}
\counterwithin{figure}{chapter}
\counterwithin{table}{chapter}
\usepackage{cmap}				% Mapear caracteres especiais no PDF
\usepackage{lmodern}			% Usa a fonte Latin Modern			
\usepackage[T1]{fontenc}		% Selecao de codigos de fonte.
\usepackage[utf8]{inputenc}		% Codificacao do documento (convers\~{a}o autom\'{a}tica dos acentos)
\usepackage{mathptmx}			% Times New Roman font
\usepackage{lastpage}			% Usado pela Ficha catalogr\'{a}fica
\usepackage{indentfirst}		% Indenta o primeiro par\'{a}grafo de cada se\c{c}\~{a}o.
\usepackage{color}				% Controle das cores
\usepackage[pdftex]{graphicx}	% Inclus\~{a}o de gr\'{a}ficos
\usepackage{epstopdf}           % Pacote que converte as figuras em eps para pdf
\usepackage{lipsum}             % Pacote que gera texto dummy
\usepackage{blindtext}          % Pacote que gera texto dummy
% ---
		
% ---
% Pacotes adicionais, usados apenas no \^{a}mbito do Modelo Can\^{o}nico do abnteX2
% ---
\usepackage{nomencl}
\usepackage{amsmath,amssymb,amsfonts,amsthm}
\usepackage{bbm}
\usepackage[chapter]{algorithm}
\usepackage{multirow}
\usepackage{rotating}
\usepackage{pdfpages}
\usepackage{gensymb}
% ---

\usepackage{algorithm}



% ---
% Pacotes de cita\c{c}\~{o}es
% ---
\usepackage[alf,abnt-etal-cite=2,abnt-etal-list=3,abnt-etal-text=emph,abnt-emphasize=bf]{abntex2cite}
\newcommand{\citet}{\citeonline}
%Cita\c{c}\~{o}es padr\~{a}o ABNT
%\citeoption{abnt-full-initials=no,abnt-url-package=url}
% ---
% Pacote de customiza\c{c}\~{a}o - Unicamp
% ---
\usepackage{unicamp}

% Pacotes necesitados por Aldo Díaz
\usepackage{enumitem}
\usepackage{array}
\usepackage{longtable}
\usepackage{mathtools}
\usepackage{tikz}
\usepackage[noend]{algpseudocode}
\usepackage{xr}

% ---
% CONFIGURA\c{C}\~{O}ES DE PACOTES
% ---

\renewcommand{\ABNTEXchapterfont}{\fontfamily{ptm}\fontseries{b}\selectfont}
\renewcommand{\ABNTEXchapterfontsize}{\large}
\renewcommand{\ABNTEXsectionfont}{\fontfamily{ptm}\fontseries{b}\selectfont}
\renewcommand{\ABNTEXsectionfontsize}{\normalsize}
\renewcommand{\ABNTEXsubsectionfont}{\fontfamily{ptm}\selectfont}
\renewcommand{\ABNTEXsubsectionfontsize}{\normalsize}

%\makeatletter
%\newcommand\thefontsize[1]{{#1 The current font size is: \f@size pt\par}}
%\makeatother


\externaldocument[1-]{cap1_intro}
\externaldocument[2-]{cap2_teoria}
\externaldocument[3-]{cap3_metodo}
\externaldocument[4-]{cap4_validation}
\externaldocument[5-]{cap5_results}
\externaldocument[6-]{cap6_conclu}

\newcolumntype{L}[1]{>{\raggedright\let\newline\\\arraybackslash\hspace{0pt}}m{#1}}
\newcolumntype{C}[1]{>{\centering\let\newline\\\arraybackslash\hspace{0pt}}m{#1}}
\newcolumntype{R}[1]{>{\raggedleft\let\newline\\\arraybackslash\hspace{0pt}}m{#1}}

\DeclarePairedDelimiter\ceil{\lceil}{\rceil}
\DeclarePairedDelimiter\floor{\lfloor}{\rfloor}
\newcommand{\Conv}{\mathop{\scalebox{1.5}{\raisebox{-0.2ex}{$\ast$}}}}
\usetikzlibrary{shapes.geometric, arrows}
%\floatname{algorithm}{Algoritmo}

\tikzstyle{startstop} = [rectangle, rounded corners, minimum width=3cm, minimum height=1cm,text centered, draw=black, fill=gray!20]
\tikzstyle{io} = [trapezium, trapezium left angle=70, trapezium right angle=110, minimum width=3cm, minimum height=1cm, text centered, text width=5cm, draw=black]
\tikzstyle{process} = [rectangle, minimum width=3cm, minimum height=1cm, text centered, text width=3cm, draw=black]
\tikzstyle{decision} = [diamond, minimum width=3cm, minimum height=1cm, text centered, draw=black]
\tikzstyle{arrow} = [thick,->,>=stealth]

% ---
\ifpdf % used graphic file format for pdflatex
  \graphicspath{{./eps/}}
  \DeclareGraphicsExtensions{.png,.jpg,.pdf}
\else  % used graphic file format for latex
  \DeclareGraphicsExtensions{.eps}
\fi

%customiza\c{c}\~{a}o do negrito em ambientes matem\'{a}ticos
\newcommand{\mb}[1]{\mathbf{#1}}
%customiza\c{c}\~{a}o de teoremas
\newtheorem{mydef}{Defini\c{c}\~{a}o}[chapter]
\newtheorem{lemm}{Lema}[chapter]
\newtheorem{theorem}{Teorema}[chapter]


% ---
% Configura\c{c}\~{o}es de apar\^{e}ncia do PDF final

% alterando o aspecto da cor azul
\definecolor{blue}{RGB}{41,5,195}

% informa\c{c}\~{o}es do PDF
\makeatletter
\hypersetup{
     	%pagebackref=true,
		pdftitle={\@title},
		pdfauthor={\@author},
    	pdfsubject={\imprimirpreambulo},
	    pdfcreator={LaTeX with abnTeX2},
		pdfkeywords={abnt}{latex}{abntex}{abntex2}{trabalho acad\^{e}mico},
		hidelinks,					% desabilita as bordas dos links
		colorlinks=false,       	% false: boxed links; true: colored links
    	linkcolor=blue,          	% color of internal links
    	citecolor=blue,        		% color of links to bibliography
    	filecolor=magenta,      	% color of file links
		urlcolor=blue,
%		linkbordercolor={1 1 1},	% set to white
		bookmarksdepth=4
}
\makeatother
% ---

% ---
% Espa\c{c}amentos entre linhas e par\'{a}grafos
% ---

% O tamanho do par\'{a}grafo \'{e} dado por:
\setlength{\parindent}{2.0cm}

% Controle do espa\c{c}amento entre um par\'{a}grafo e outro:
\setlength{\parskip}{0.2cm}  % tente tamb\'{e}m \onelineskip

% ---
% Informacoes de dados para CAPA e FOLHA DE ROSTO
% ---
\titulo{Sistema de Decolagem e Pouso de Precisão com Quadrotor Autônomo Usando Detecção de Tags Aruco e Redes Neurais Convolucionais}
\autor{Alan Ferreira Pinheiro Tavares}
\local{Campinas}
\data{2018}
\orientador{Prof. Dr. Paulo Roberto Gardel Kurka}
%\coorientador[Co-orientador]{Prof. Dr. Co-orientador}
\instituicao{%
    UNIVERSIDADE ESTADUAL DE CAMPINAS
    \par
    Faculdade de Engenharia Mecânica	
    }
%\tipotrabalho{Tese (Doutorado)}
%% O preambulo deve conter o tipo do trabalho, o objetivo, o nome da institui\c{c}\~{a}o e a \'{a}rea de concentra\c{c}\~{a}o
%\preambulo{Tese apresentada \`{a} Faculdade de Engenharia El\'{e}trica e de Computa\c{c}\~{a}o da Universidade Estadual de Campinas como parte dos requisitos exigidos para a obten\c{c}\~{a}o do t\'{\i}tulo de Doutor em Engenharia El\'{e}trica, na \'{A}rea de Engenharia de Computa\c{c}\~{a}o.}
\tipotrabalho{Tese (Doutorado)}
\preambulo{Tese de Doutorado apresentada à Faculdade de Engenharia Mecânica da Universidade Estadual de Campinas como parte dos requisitos exigidos para a obtenção do título de Doutor em Mecatrônica.}

% --- 

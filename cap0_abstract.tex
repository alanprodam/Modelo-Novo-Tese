In 2009, over 56 million liters of fuel were used to overcome friction in combustion engines, being the bearings alone accountable for nearly a quarter of this loss. This research aims to numerically model a lightweight crankshaft bearing and to study the effects that surface textures can have on its performance, aiming to reduce friction in this component.  A literature review was performed, covering the surface texture treatments, lubrication regimes and aspects of rough contact and temperature. A computer program was developed to study the behavior of a dynamically loaded bearing, considering roughness effects on lubrication, asperity contact, global thermal effects and surface texturing. Simulations were performed for the bearing of a lightweight crankshaft, considering different texture designs on its surface. The results showed that it is possible to reduce the mean and maximum hydrodynamic friction coefficient of these bearings, but some texture designs may lead to an increase in the fluid pressure, specially when the thermal effects are take into account.
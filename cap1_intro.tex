\chapter{Introdução}

\section{Motivação}

O uso de Veículos Autônomos Aéreos Não Tripulados (VAANTs), também conhecidos como \textit{drones} ou quadrotores autônomos, vêm ganhando notoriedade nas últimas décadas devido à sua alta versatilidade, manobrabilidade, avanço tecnológico computacional, seguido também de aplicações inovadoras em quase todos os setores, sejam elas com propósito militar, de pesquisa, comércio, vigilância ou de lazer. Justifica-se também este interesse por conta do crescente número de fabricantes, redução de preços, melhoria da qualidade dos componentes e sensores envolvidos, demandando assim técnicas mais robustas para garantir a segurança das aeronaves \cite{Santos2014}.

Segundo definições de \citet{Duan2010}, VAANT é um tipo de sistema muito complexo que integra diferentes componentes de hardware, como o Sistema de Posicionamento Global (GPS), Unidade de Gerenciamento Inercial (IMU), controlador e diferentes componentes de software, como processamento de imagem e planejamento de trajetória. \citet{Costa2012} destaca a importância da IMU no sistema de localização dos VAANTs que integra bússola, barômetro, sonares e sensor GPS. Tanto o IMU quanto a bússola indicam a atitude; Barômetro e sonar podem ser usados para leitura de altitude, e o GPS para obter dados de posicionamento e também altitude. 

\subsection{Problematização da Descolagem e Pouso}

Contudo, estes sensores de navegação podem apresentar erros da ordem de metros e de forma acumulativa, principalmente quando se diz a respeito de pousos ou decolagens utilizando GPS próximos a prédios, redes elétricas ou montanhas que bloqueiam ou anulam completamente o sinal de GPS. Este fato deprecia a qualidade, confiabilidade e custo da missão, visto que se faz necessário a aquisição de sensores com altíssima precisão e com blindagem eletrostática específica, resultando em custo elevado. Logo, a visão computacional é uma excelente solução de baixo custo e alto retorno de informações espaciais para o controle de VAANTs.
 
Tal solução é apresentada através das câmeras de sensor exteroceptivo ou monocular, que além de atribuir precisão em navegação possui a vantagem de ser um sensor passivo, de baixa sensibilidade a ruídos e independente do sinal de GPS, permitindo voo em ambientes onde esse sinal não possui boa qualidade \cite{GUO2014}. Pode-se citar algorítimos de código aberto como de \citet{Forster2014} e \citet{Raul2016}, que trazem robustas soluções de navegação por visual monocular através de uma abordagem semi-direta que elimina a necessidade de extração de recursos caros e técnicas de correspondência robustas para estimativa de posição e orientação.

%===================================================================================================

\subsection{Aplicação de Drones em Serviçoes de encomendas \textit{(delivery)}}

Uma das aplicações dos estudos desta tecnologia está presente no setor de encomendas \textit{(delivery)}, onde são utilizados pelo serviço de entregas em algumas regiões como nos Estados Unidos, Austrália e Suíça. Um dos exemplos é o serviço \textit{Prime Air} da companhia \textit{Amazon}, que promete atender a encomendas de até 5 \textit{libras} em meia hora, com veículos cobertos de redundâncias "\textit{sense and avoid}" para garantir a segurança durante o tráfego em rotas especiais. Na Suíça estão sendo testados \textit{drones} para entregas de encomendas pesando até um quilograma por distâncias que se estendem a dez quilômetros, segundo notícias de Agosto de 2016 \cite{Vidal2016}.

Segundo notícias de Maio de 2018. O Brasil já realizou a primeira simulação de \textit{delivery}, que ocorreu em São Paulo com autorização do Departamento de Controle do Espaço Aéreo (DECEA). Com o objetivo de entregar medicamentos, a ação utilizou um hexacóptero e um programa desenvolvidos pela marca nacional \textit{SMX Systems}. Apesar de ser considerada a primeira simulação oficial, esta não foi a primeira vez em que uma entrega é feita com \textit{drone} no país, já que em 2014 houve um \textit{delivery} de pizza no município de Santo André (SP) \cite{Techtudo2018}.


\subsection{Problematização Linhas de Transmissão}

\subsection{Sugestão da Solução}

Com base nos argumentos descritos acima, verifica-se a motivação da tesa no âmbito de desenvolver e simular um sistema de decolagem e pouso de precisão autônoma para um VAANT do tipo quadrotor. A tese visa realizar o controle do sistema utilizando somente técnicas de visão computacional simuladas via V-REP usando um quadrotor com câmera monocular. O algorítimo inicialmente detecta o local de decolagem/pouso via marcador fiducial \textit{tag} ArUco, em seguida retorna a posição do quadrotor em relação ao marcador, contornando erros inerentes a sensores de posicionamento, como IMU e GPS. A proposta baseia-se em obter um algorítimo que utilize a Rede Neural Convolucional (CNN) de forma a complementar a detecção da \textit{tag} ArUco clássica da maneira mais eficaz. Deseja-se criar dados de treinamento híbridos através de dados sintéticos e reais de forma a superar a limitações de casos específicos como detecção imprecisa a longa distância, oclusão parcial ou uso de câmeras com alta distorção. 

Verifica-se a utilização de ArUco pelo fato de ser uma biblioteca de código aberto para detecção de marcadores fiduciais. Segundo \citet{Aruco2014}, a biblioteca permite estimar a pose de uma câmera a partir de um ou vários marcadores que são analisados em paralelo, onde, cria-se uma imagem canônica no nível da pirâmide que alcança o melhor equilíbrio entre qualidade e tamanho, sendo atualmente uma das estratégias mais investigas para uso como marcador de pouso e rastreamento com VAANTs, além de dispor de criação de marcadores personalizados. 

%Contudo, tem-se ainda o desafio de identificação a longo alcance, visto que o marcador fiducial tem um limite de tamanho por distancia de identificação segura. Ou seja, para alturas elevadas acima de 30 metros se faz necessário o uso de uma segunda abordagem, um algoritmo de odometria visual monocular que mesmo sem a identificação do marcador, guie o \textit{drone} próximo a altura de identificação segura. 

%===================================================================================================

%Logo o trabalho traz a proposta de se utilizar a biblioteca ArUco  em conjunto com o algorítimo de \citet{Forster2014}, uma técnica já consolidada e precisa que segundo o autor a mais rápida dos métodos atuais de última geração. Tem-se como vantagem, a abordagem semi-direta que elimina a necessidade de extração de recursos caros e técnicas de correspondência robustas para estimativa de movimento. O algorítimo opera diretamente nas intensidades de pixel, o que resulta em precisão de subpixel em altas taxas de quadros. Chamado pelo autor como abordagem de \textit{Odometria Visual Semi-Direta} (SVO), já foi testado em veículos micro-aéreos e lançado como \textit{software} de código aberto.

%Em relação ao desafio de custo elevado versos desempenho computacional, conta-se com a estratégia de embarcar o \textit{software} desenvolvido em um \textit{hardware} de dispositivo \textit{smartphones} via plataforma \textit{Android} com o objetivo obter desempenho computacional bom ou similar a computadores convencionais quando utilizados para o mesmo objetivo. As vantagens de se embarcar uma interface \textit{Android} é facilidade de comunicação e uso dos periféricos já embarcados no dispositivo, além dos ganhos em custo benefício entre processamento e consumo de energia.

%Por fim, serão realizados testes experimentais que verifiquem o desempenho do sistema desenvolvido frente a 2 (duas) abordagens utilizadas para pouso de precisão: 1.PX4Flow via técnica Optical Flow; 2.IR-LOCK via câmera PIXY. Estima-se o uso do sistema em uma aplicação real em missões de entregas (\textit{deliveries}) de medicamento na cidade do Porto-Portugal vinculada a empresa \textit{Connec Robotics}. As mesmas técnicas utilizadas para o pouso bodem ser replicadas para a decolagem que também apresenta grandes problemáticas quando guiada somente por GPS. 

\subsection{Objetivo}

O presente trabalho tem como objetivo principal desenvolver e simular um sistema de decolagem e pouso de precisão autônoma para um VAANT do tipo quadrotor. A tese visa realizar o controle do sistema utilizando somente técnicas de visão computacional simuladas via V-REP usando um quadrotor com câmera monocular. O algorítimo inicialmente detecta o local de decolagem/pouso via marcador fiducial \textit{tag} ArUco, em seguida retorna a posição do quadrotor em relação ao marcador, contornando erros inerentes a sensores de posicionamento, como IMU e GPS. A proposta baseia-se em obter um algorítimo que utilize a Rede Neural Convolucional (CNN) de forma a complementar a detecção da \textit{tag} ArUco clássica da maneira mais eficaz. Deseja-se criar dados de treinamento híbridos através de dados sintéticos e reais de forma a superar a limitações de casos específicos como detecção imprecisa a longa distância, oclusão parcial ou uso de câmeras com alta distorção. . A metodologia adotada demanda as seguintes etapas:

\begin{itemize}
  \item Realizar levantamento bibliográfico sobre os principais conceitos e desafios de pesquisa referentes a pouso e decolagem de precisão autônoma utilizando veículos quadrotores;
  
  \item Embarcar bibliotecas de OpenCV, ArUco e \textit{Tensorflow}  vinculados a linguagem de programação C/C++ ou \textit{Python} em uma única IDE;
  
  \item Estruturar um dicionário de marcadores fiduciais que apresente bons desempenhos nos quesitos de identificação precisa do marcador e confiabilidade de posição;
  
  \item Desenvolver inicialmente um algorítimo que tenha como saída a posição do marcador de pouso em relação ao quadrotor, baseada somente na técnica de detecção via \textit{tag} ArUco;
  
  \item Utilizar o Simulador V-REP para integrar a comunicação de controle do quadrotor com o algorítimo desenvolvido;
  
   \item Realizar a aquisição de dados sintéticos de treinamento de rede neural através de ensaios simulados de decolagem e pouso com quadrotores via V-REP;
  
  \item Realizar a aquisição de dados reais de treinamento de rede neural através de ensaios reais de decolagem e pouso com quadrotores via gravações de \textit{datasets};
   
   \item  Realizar treinamento dos dados obtidos para identificar o marcador desejado utilizando técnica de Rede Neural Convolucional (CNN);
   
   \item Implementar e simular no V-REP o algorítimo desenvolvido (CNN) de forma a complementar a detecção da \textit{tag} ArUco clássica da maneira mais eficaz.
   
   %\item Coletar dados estatísticos dos ensaios simulados via V-REP e analisar o desempenho do sistema autônomo projetado;
   
    \item Determinar os melhores parâmetros visando eficiência na identificação de marcadores a longa distância, onclusões parciais e distorção de câmera;
    
\end{itemize}

%===================================================================================================

\section{Estado da Arte}

%\subsection{Visão Geral: Veículos Autônomos Aéreos Não Tripulados VAANTS}

	%\subsubsection{Conceitos e Fundamentos}

	%\subsubsection{Desafios de Controle e Navegação em Pouso/Decolagem}

	%\subsubsection{Aplicações Comerciais}

\subsection{Trabalhos Relacionado: Descolagem e Pouso Autônomo}

	%\textbf{Revisão Literária das Principais Técnicas de Decolagem e Pouso}

	%\textbf{Revisão Literária do Uso de Marcadores Fiduciais como \textit{Landmarks}}

	%\textbf{Revisão Literária das Técnicas de Redes Neurais Usadas para Aperfeiçoar a Detecção de \textit{Tags} ArUco}
	
	%Neste tópico, serão abordados os principais desafios referentes às técnicas existentes de decolagem e pouso de precisão com um quadrotor sem o uso de sensores de posicionamento, como IMU e GPS. O entendimento dessas técnicas são abordados através do levantamento bibliográfico e análise comparativa da literatura recente, sendo o principal objetivo deste tópico.
	
	A revisão de literatura sugere que a maioria das implementações de técnicas desenvolvidas disponíveis podem ser classificadas em 2 (duas) categorias: A primeira com sistemas que fazem uso de sensores de imagem e a segunda com sistemas cujas capacidades de navegação são baseadas em sensores ativos, como lasers e sonares em casos de ambientes internos e IMU e GPS em casos de ambiente externo.
	
    abordagem que utiliza marcadores fiduciais clássicos
    abordagem que utiliza PX4Flow via técnica Optical Flow e IR-LOCK via câmera PIXY. 
    
	A revisão literária sobre as principais técnicas de decolagem e pouso de precisão para VANTs do tipo quadrotor usadas nesta tese é resumida através da Tabela~\ref{qd:estado-da-arte}, onde, destaca-se a principal abordagem e contribuição científica de cada referência. Em seguida, realiza-se uma análise comparativa destes trabalhos em relação ao trabalho proposto e os principais desafios superados por cada técnica. 
	
    \begin{table}[h]
    \centering
	\caption{Resumo das principais abordagens e contribuições científicas relacionadas à decolagem e pouso de VANTs usando somente visão computacional.}
	%\vspace{0.5cm}
		\begin{tabular}{ccc}
		   
			\hline
			\multirow{3}{*}{\textbf{Abordagem}} & \multirow{3}{*}{\textbf{Contribuição}} & \multirow{3}{*}{\textbf{Referências}} \\
			&                                  &                         &                      \\ \hline
			\multirow{3}{*}{Pouso Autônomo em} \\ \multirow{4}{*}{Plataforma Móvel} & \multirow{2}{*}{Utilização de Navegação Monocular} & \multirow{2}{*}{\citet{Gilberto2016}} \\
			&                                  &                         &                       \\ \hline
			\multirow{3}{*}{landing}   & \multirow{3}{*}{pouso de precisão com Rqcode}      & \multirow{3}{*}{\citeonline{Forster2014}} \\
			&                                  &                         &                       \\ \hline
			\multirow{3}{*}{aruco}     & \multirow{3}{*}{novo modelo de Aruco}              & \multirow{2}{*}{\citet{Forster2014}} \\
			&                                  &                         &                        \\ \hline
			\multirow{3}{*}{aruco}     & \multirow{3}{*}{criação de dicionarios Aruco}      & \multirow{2}{*}{\citet{Jayatilleke2013}} \\
			&                                  &                         &                         \\ \hline
			\multirow{3}{*}{redes cnn} & \multirow{3}{*}{Inteligencia artificial em marcadores} & \multirow{2}{*}{\citet{Faigl2013}} \\
			&                                  &                         &                          \\ \hline
			\multirow{3}{*}{redes cnn} & \multirow{3}{*}{CNN aplicado em aruco}             & \multirow{2}{*}{\citet{Pestana2016}} \\
			&                                  &                         &                           \\ \hline
			\multirow{3}{*}{redes cnn} & \multirow{3}{*}{CNN aplicado em aruco}             & \multirow{2}{*}{\citet{Salinas2013}} \\
			&                                  &                         &                           \\ \hline
		\end{tabular}

	\label{qd:estado-da-arte}
\end{table}
    
    O trabalho de \citet{Gilberto2016} traz uma abordagem de pouso autônomo em uma plataforma móvel, destaca-se a utilização de técnicas de odometria monocular + estimação da plataforma através de uma segunda câmera, onde os mesmo já sabem o modelo cinemática da base móvel, aplicou-se um filtro de calma entre o modelo cinemático, IMU e visão para aumentar a frequência de amostragem
    
    A principal motivação de usar sensores de imagem é seu peso leve e baixo consumo de energia. Para este fim, vários abordagens utilizaram marcadores visuais para simplificar a problema de localização \cite{Jayatilleke2013}, \cite{Faigl2013}, \cite{Pestana2016} e \cite{Salinas2013}.

%===================================================================================================

\subsection{Contribuição Científica: Algorítimo de Decolagem e Pouso Autônomo}

	\subsubsection{Dicionário de Marcadores para Pouso/Decolagem de Identificação Única}

	\subsubsection{Utilização da (CNN) para Complementar a detecção da \textit{tag} ArUco de forma Eficaz}

	\subsubsection{Desenvolvimento de um Sistema de Baixo Custo para Decolagem e Pouso Autônomo de Quadrotores sem uso de GPS ou IMU}

\section{Estruturação da Tese}

Os capítulos seguintes descrevem o trabalho da seguinte forma: O capítulo 2 (dois) traz o estado da arte, descrevendo uma visão geral sobre os VAANTs, os trabalhos relacionas e por fim a contribuição cientifica da pesquisa; O capítulo 3 (três) aborda sobre o sistema de pouso e decolagem desenvolvido, bem como as técnicas matemáticas para localização da aeronave no ambiente, descreve as linguagens e bibliotecas de programação empregadas no desenvolvimento do algoritmo e a arquitetura para controle e comunicação entre as diversas partes do projeto; O capítulo 4 (quatro) mostra os resultados e discussões de simulação e testes experimentais dos processo na execução do pouso e decolagem; Por fim, o capítulo 5 (cinco) traz todas as conclusões do trabalho realizado e sugestões de trabalhos futuros.




